\thusetup{
  %******************************
  % 注意:
  %   1. 配置里面不要出现空行
  %   2. 不需要的配置信息可以删除
  %******************************
  %
  %=====
  % 秘级
  %=====
  secretlevel={秘密},
  secretyear={10},
  %
  %=========
  % 中文信息
  %=========
  ctitle={FlexTouch: 一种基于电容屏与柔性导电材料的可扩展式触摸交互技术},
  cdegree={工学硕士},
  cdepartment={全球创新学院},
  cmajor={数据科学和信息技术},
  cauthor={周建宇},
  csupervisor={喻\hspace{1.1cm} 纯 副研究员},
  cassosupervisor={王运涛 助理研究员}, % 副指导老师
%   ccosupervisor={某某某教授}, % 联合指导老师
  % 日期自动使用当前时间,若需指定按如下方式修改:
  cdate={二〇一九年十二月},
  %
  % 博士后专有部分
%   catalognumber     = {分类号},  % 可以留空
%   udc               = {UDC},  % 可以留空
%   id                = {编号},  % 可以留空: id={},
%   cfirstdiscipline  = {计算机科学与技术},  % 流动站(一级学科)名称
%   cseconddiscipline = {系统结构},  % 专 业(二级学科)名称
%   postdoctordate    = {2009 年 7 月——2011 年 7 月},  % 工作完成日期
%   postdocstartdate  = {2009 年 7 月 1 日},  % 研究工作起始时间
%   postdocenddate    = {2011 年 7 月 1 日},  % 研究工作期满时间
  %
  %=========
  % 英文信息
  %=========
  etitle={FlexTouch: Extending Interaction Sensing beyond TouchScreens using Flexible and Conductive Materials},
  % 这块比较复杂,需要分情况讨论:
  % 1. 学术型硕士
  %    edegree:必须为Master of Arts或Master of Science(注意大小写)
  %             “哲学、文学、历史学、法学、教育学、艺术学门类,公共管理学科
  %              填写Master of Arts,其它填写Master of Science”
  %    emajor:“获得一级学科授权的学科填写一级学科名称,其它填写二级学科名称”
  % 2. 专业型硕士
  %    edegree:“填写专业学位英文名称全称”
  %    emajor:“工程硕士填写工程领域,其它专业学位不填写此项”
  % 3. 学术型博士
  %    edegree:Doctor of Philosophy(注意大小写)
  %    emajor:“获得一级学科授权的学科填写一级学科名称,其它填写二级学科名称”
  % 4. 专业型博士
  %    edegree:“填写专业学位英文名称全称”
  %    emajor:不填写此项
  edegree={Master of Science },
  emajor={Data Science and Information Technology},
  eauthor={Jianyu Zhou},
  esupervisor={Associate Professor Yu Chun},
  eassosupervisor={Assistant Professor Wang Yuntao},
  % 日期自动生成,若需指定按如下方式修改:
  edate={December, 2019},
  %
  % 关键词用“英文逗号”分割
  ckeywords={电容感应, 交互接口, 低成本制造, 2D电路设计, 大规模交互},
  ekeywords={Capacitive Sensing, Touch Interface, Low-cost Fabrication, 2D Circuit Pattern, Large-Scale Interaction},
}

% 定义中英文摘要和关键字
\begin{cabstract}
  目前基于感应电容的触摸屏交互已得到大规模应用,极大改变了人机交互模式,使人机交互更自然更、高效。然而,手机、平板电脑等电容屏设备的交互面积仍十分有限,很多情况下无法满足人们对更大规模的触摸交互需求,如桌面交互。而如果电容屏过大,第一不方便携带,第二制造成本也会大幅上升。如何在不改变现有触摸屏设备的基础上增加交互面积,成为了当前人机交互领域的一个热点研究方向,而本文的工作则是研究如何利用基于以手机为代表的智能交互设备的计算能力和电容特性,以低成本的方式将人机交互的平面从屏幕表面解放出来,扩展至计算设备周围更大的平面,满足用户大规模交互需求。

  本文提出了一种新的技术,使用廉价易得的导电材料即可在不改变交互设备本身的前提下大大增加交互面积,我们称为FlexTouch。我们利用柔性导电材料如铜铂线,将触摸屏的边缘与交互平面通过定制化的2D纸电路相连接,从而实现大规模交互,这种交互技术无需对已有设备做任何改造。我们通过实验证明了FlexTouch可应用于多种导电材料,包括铜箔胶带(copper foil tape)、纳米银墨(silver nanoparticle ink)、氧化铟锡薄膜(ITO film)和碳涂料(carbon paint),同时也适用于多种制造工艺,包括剪切技术、表面涂层技术和油墨喷印技术。我们的一系列研究和实验表明,FlexTouch可识别4米以内的人体触摸和2米以内的物体感知。最后,基于我们的识别原理和实验结果,我们展示了FlexTouc在不同场景下应用的可行性,例如可以在瑜伽垫上识别人体姿势,在桌面上检测人与水杯等物体的状态,以及在跑步机上检测人体的运动情况等。 Flextouch相比于之前的技术主要有以下优势:1)交互设备无需外部供电,可直接利用手机等智能设备的电源;2)改造简单,几乎无需改造交互平面,只需通过设计好的交互接口使手机/平板电脑与交互平面连接即可;3)改造成本低,实验所使用的柔性导电材料廉价易得。4)同时支持一维和二维交互,增加了有限面积内的感应密度。
    
  本文重点研究了1)不同导电材料对电容信号的影响,给出了不同材料的交互面积上限,为应用设计给出了理论指导;2)引入接地线后对电容感应信号的影响,进一步增强了信号识别率,为大范围交互提供了可能;3)多种不同的大范围纸电路设计和相应的交互应用场景,为更多不同模态的交互模式提供了参考;4)各种示例应用的在鲁棒性、准确性等方面的实际测试表现,证明了技术的可行性。
    
\end{cabstract}

% 如果习惯关键字跟在摘要文字后面,可以用直接命令来设置,如下:
% \ckeywords{\TeX, \LaTeX, CJK, 模板, 论文}

\begin{eabstract}
  Capacitive Touchscreens has been widely adopted in rencent ten years, greatly improved the effectiveness of human-computer interaction, enabling more natrual, more efficient interaction patterns. However, the interaction area of smartphone and tablet is still limited to the area of screen surface, which in many cases cannot meet the demands of large-scale interaction, such as table-based interaction. A straitforward solution to this issue is naively increase the area of touchscreen. However, there are at least two drawbacks: 1) Portability, 2) Cost Effectiveness. How to fully leverage existing smart device to enalbie large-scale interaction without modifications of smart device itself has become a popular research topic in Human-Computer Interaction.

  In this paper, we present FlexTouch, a technique that enables large-scale interaction sensing beyond the spatial constraints of capacitive touchscreens using passive low-cost conductive materials. This is achieved by customizing 2D circuit-like patterns with an array of conductive strips that can be easily attached to the sensing nodes on the edge of the touchscreen. This retrofit requires no hardware modification to the device. FlexTouch is compatible with various conductive materials (copper foil tape, silver nanoparticle ink, ITO frame, and carbon paint), as well as fabrication methods (cutting, coating, and ink-jet printing). We demonstrate that FlexTouch can support long-range touch sensing for up to 4 meters and everyday object presence detection for up to 2 meters. Finally, we show the versatility and feasibility of FlexTouch through applications such as whole-body posture recognition, human-object interaction as well as enhanced fitness training experiences. FlexTouch has four advantages: 1) No need of external power source. 2) Simple fabrication. 3) Low cost. 4) 2D patterns that enables more sensing nodes in limited area.  
  
  Our major contributions are 1) Design-scope study of interaction with different materials. 2) Analysis of grouding strip effect. 3) Example applications and scenarios. 4) Evaluation on robustness and accuracy of different scenarios.
\end{eabstract}

% \ekeywords{\TeX, \LaTeX, CJK, template, thesis}
