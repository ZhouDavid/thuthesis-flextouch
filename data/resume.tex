\begin{resume}

  \resumeitem{Education}

Sep 2013–Jul 2017, Bachelor of Science in Engineering in Control Science and Technology, Tsinghua University.

Sep 2017-Jul 2019, Master of Science in Technology Innovation, University of Washington.

Sep 2017-Jul 2019, Master of Science in Engineering in Data Science and Information Technology, Tsinghua University.

\resumeitem{Awards}
Jun 2018, Second Prize of the National University Blockchain Competition in North China.

Oct 2017, Third Prize of National College Students Innovation and Entrepreneurship Competition.


%   \researchitem{Awards} % 发表的和录用的合在一起

%   % 1. 已经刊载的学术论文(本人是第一作者,或者导师为第一作者本人是第二作者)
%   \begin{publications}
%     \item 杨轶, 张宁欣, 任天令, 等. 硅基铁电微声学器件中薄膜残余应力的研究. 中国机
%       械工程, 2005, 16(14):1289-1291. (EI 收录, 检索号:0534931 2907.)
%   \end{publications}

%   % 2. 尚未刊载,但已经接到正式录用函的学术论文(本人为第一作者,或者
%   %    导师为第一作者本人是第二作者)。
%   \begin{publications}[before=\publicationskip,after=\publicationskip]
%     \item Yang Y, Ren T L, Zhu Y P, et al. PMUTs for handwriting recognition. In
%       press. (已被 Integrated Ferroelectrics 录用. SCI 源刊.)
%   \end{publications}

%   % 3. 其他学术论文。可列出除上述两种情况以外的其他学术论文,但必须是
%   %    已经刊载或者收到正式录用函的论文。
%   \begin{publications}
%     \item Wu X M, Yang Y, Cai J, et al. Measurements of ferroelectric MEMS
%       microphones. Integrated Ferroelectrics, 2005, 69:417-429. (SCI 收录, 检索号
%       :896KM)
%     \item 贾泽, 杨轶, 陈兢, 等. 用于压电和电容微麦克风的体硅腐蚀相关研究. 压电与声
%       光, 2006, 28(1):117-119. (EI 收录, 检索号:06129773469)
%     \item 伍晓明, 杨轶, 张宁欣, 等. 基于MEMS技术的集成铁电硅微麦克风. 中国集成电路,
%       2003, 53:59-61.
%   \end{publications}

%   \researchitem{研究成果} % 有就写,没有就删除
%   \begin{achievements}
%     \item 任天令, 杨轶, 朱一平, 等. 硅基铁电微声学传感器畴极化区域控制和电极连接的
%       方法: 中国, CN1602118A. (中国专利公开号)
%   \end{achievements}

\end{resume}
